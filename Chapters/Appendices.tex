\begin{appendices}

\section{Data types} \label{sec:DataTypes}

\subsection*{General data types}

\vspace{4pt}
\begin{center}

	\small
	\begin{tabular}{| >{\centering} m{40pt} | >{\centering\arraybackslash} m{50pt} | m{240pt} |}
		
		\hline
		\textbf{Type} & \bigcell{c}{\textbf{Size} \\ \textbf{(Bytes)}} & \textbf{Description}\\ \hline \hline
		
		int & 4 & Signed integer in little-endian. \\ \hline
		uint & 4 & Unsigned integer in little-endian. \\ \hline
		uint8\_t & 1 & Unsigned integer. \\ \hline
		uint16\_t & 2 & Unsigned integer in little-endian. \\ \hline
		uint32\_t & 4 & Unsigned integer in little-endian. \\ \hline
		uint64\_t & 8 & Unsigned integer in little-endian. \\ \hline
		uint160 & 20 & Unsigned integer array uint32\_t[] of size 5. \newline Used for storing RIPEMD160 hashes as a byte array.\\ \hline
		uint256 & 32 & Unsigned integer array uint32\_t[] of size 8. \newline Used for storing SHA256 hashes as a byte array.\\ \hline
	\end{tabular}
\end{center}
\vspace{10pt}

\subsection*{Variable length integers (VarInt)}
Integers in Bitcoin can be encoded depending on the value in order to save space. Variable length integers always precede vectors of a type of data that may vary in length. An overview of the different variable length integers is depicted below.

\renewcommand{\arraystretch}{1.25}
\begin{center}
	\small
	\begin{tabular}{| >{\centering} m{60pt} | >{\centering} m{50pt} | m{180pt} |}
		
		\hline
		\textbf{Value interval} & \bigcell{c}{\textbf{Size} \\ \textbf{(Bytes)}} & \textbf{Format} \\ \hline \hline
	    
	    $\left[{0,2^{8}-3}\right)$ & 1 & uint8\_t \\ \hline
		$\left[2^{8}-3, 2^{16}\right)$ & 3 & 0xFD followed by the value as uint16\_t \\ \hline
		$\left[2^{16}, 2^{32}\right)$ & 5 & 0xFE followed by the value as uint32\_t \\ \hline
		$\left[2^{32}, 2^{64}\right)$ & 9 & 0xFF followed by the value as uint64\_t \\ \hline
	\end{tabular}
\end{center}
\renewcommand{\arraystretch}{1}

\clearpage
\section{Formulas} \label{sec:Formulas}

\subsubsection*{Transaction Fees}
The transaction fees \emph{TxFee} in Satoshi are calculated from the transaction size \emph{TxSize} in bytes and the transaction fee rate \emph{TxFeeRate} in Satoshi per kB as follows:

\begin{equation}
TxFee = TxFeeRate \cdot \ceil[\bigg]{\dfrac{TxSize}{1000}}
\end{equation}

\noindent
The transaction fee rate currently lies at 10000 Satoshi per kB and can, if desired, be changed by the user. Note, however, that if it lies below the minimum transaction fee rate of 1000 Satoshi per kB, then it will neither be relayed nor mined.

\subsubsection*{Dust Transactions}
A transaction is defined as ``dust'', if any of the transaction outputs spends more than 1/3rd of its value in transaction fees. More precisely, a transaction is considered ``dust'' if any of its transaction outputs satisfies the inequality

\begin{equation}
\dfrac{\dfrac{MinTxFeeRate}{1000} \cdot (TxOutSize + 148)}{nValue} > \dfrac{1}{3}
\end{equation}
\vspace{0pt}

\noindent
where \emph{nValue} is the transaction output value, \emph{TxOutSize} is the transaction output size in bytes and \emph{MinTxFeeRate} is the minimum transaction fee rate in Satoshi per kB.

\subsubsection*{Minimum Spending Amount}
It follows from the dust transaction rule that every transaction output has a minimum spending amount defined by

\begin{equation}
MinValue = 3 \cdot \dfrac{MinTxFeeRate}{1000} \cdot (TxOutSize + 148)
\end{equation}

\noindent
Given that the default minimum transaction fee rate currently lies at 1000 Satoshi per kB and that the size of a typical transaction output is 34 bytes, the resulting minimum spending amount is 546 Satoshi.

\end{appendices}