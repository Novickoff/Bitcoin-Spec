\begin{appendices}

\section{Data types} \label{sec:DataTypes}

\subsection*{General data types}

\vspace{4pt}
\begin{center}
	\begin{tabularx}{\textwidth}{| c | c | X |}
		\hline
		\textbf{Type} & \textbf{Size (bytes)} & \textbf{Description} \\ \hline \hline
		int & 4 & Signed integer in little-endian. \\ \hline
		uint & 4 & Unsigned integer in little-endian. \\ \hline
		uint8\_t & 1 & Unsigned integer. \\ \hline
		uint16\_t & 2 & Unsigned integer in little-endian. \\ \hline
		uint32\_t & 4 & Unsigned integer in little-endian. \\ \hline
		uint64\_t & 8 & Unsigned integer in little-endian. \\ \hline
		uint160 & 20 & Unsigned integer array uint32\_t[] of size 5. \newline Used for storing RIPEMD160 hashes as a byte array.\\ \hline
		uint256 & 32 & Unsigned integer array uint32\_t[] of size 8. \newline Used for storing SHA256 hashes as a byte array.\\ \hline
	\end{tabularx}
\end{center}
\vspace{10pt}

\subsection*{Variable length integers (VarInt)}
Integers in Bitcoin can be encoded depending on the value in order to save space. Variable length integers always precede vectors of a type of data that may vary in length. An overview of the different variable length integes is depicted below.

\begin{center}
\begin{tabular}{| C{0.3\linewidth} | C{0.1\linewidth} | C{0.5\linewidth} |}
	\hline
	\textbf{Value interval} & \bigcell{c}{\textbf{Size} \\ \textbf{(bytes)}} & \textbf{Format} \\ \hline \hline
    $\left[{0,2^{8}-3}\right)$ & 1 & uint8\_t \\ \hline
	$\left[2^{8}-3, 2^{16}\right)$ & 3 & 0xFD followed by the value as uint16\_t \\ \hline
	$\left[2^{16}, 2^{32}\right)$ & 5 & 0xFE followed by the value as uint32\_t \\ \hline
	$\left[2^{32}, 2^{64}\right)$ & 9 & 0xFF followed by the value as uint64\_t \\ \hline
\end{tabular}
\end{center}


\clearpage
\section{Calculations} \label{sec:Calculations}
\subsubsection*{Transaction Fees}~\\
The transaction fees \emph{TxFee} in Satoshi are calculated from the transaction size \emph{TxSize} in bytes and the transaction fee rate \emph{TxFeeRate} in Satoshi per kB as follows:

\begin{equation}
TxFee = \dfrac{TxFeeRate}{1000} * \ceil[\bigg]{\dfrac{TxSize}{1024}}
\end{equation}

\subsubsection*{Dust Transactions}~\\
A transaction is defined as ``dust'', if any of the transaction outputs spends more than 1/3rd of its value in transaction fees. More precisely, a transaction is considered ``dust'' if any of its transaction outputs satisfies the inequality

\begin{equation}
\dfrac{\dfrac{TxFeeRate}{1000}*(TxOutSize + 148)}{nValue} > \dfrac{1}{3}
\end{equation}
\vspace{0pt}

\noindent
where \emph{nValue} is the transaction output value, \emph{TxOutSize} is the transaction output size in bytes and \emph{TxFeeRate} is the transaction fee rate in Satoshi per kB. Given that the default transaction fee rate currently lies at 1000 Satoshi per kB and that the size of a typical transaction output is 34 bytes, the resulting minimal transaction output value is 546 Satoshi.


%\clearpage
%\section{Script Commands} \label{sec:ScriptCommands}
%
%\subsection*{Constants}
%\begin{center}
%	\begin{tabular}{| C{0.225\linewidth} | C{0.125\linewidth} |
%						>{\em}C{0.125\linewidth} | >{\em}C{0.125\linewidth} | C{0.35\linewidth} |}
%		\hline
%		\textbf{Word} & \textbf{Opcode} & \textbf{Input} & \textbf{Output} & \textbf{Description} \\ \hline \hline
%		
%		OP\_0, OP\_FALSE & 0x00 &
%		None & Empty value &
%		An empty array of bytes is pushed onto the stack.\\ \hline
%	
%		N/A	& 0x01-0x4B	&
%		(Special) &	Data &
%		The next \emph{opcode} bytes is data to be pushed onto the stack.\\ \hline
%		
%		OP\_PUSHDATA1 &	0x4C &
%		(Special) & Data &
%		The next byte contains the number of bytes to be pushed onto the stack.\\ \hline
%		
%		OP\_PUSHDATA2 & 0x4D &
%		(Special) & Data &
%		The next two bytes contain the number of bytes to be pushed onto the stack.\\ \hline
%		
%		OP\_PUSHDATA4 &	0x4E &
%		(Special) & Data &
%		The next four bytes contain the number of bytes to be pushed onto the stack.\\ \hline
%		
%		OP\_1NEGATE	& 0x4F &
%		None & -1 &
%		The number -1 is pushed onto the stack.\\ \hline
%		
%		OP\_1, OP\_TRUE & 0x51 &
%		None & 1 &
%		The number 1 is pushed onto the stack.\\ \hline
%		
%		OP\_2-OP\_16 & 0x52-0x60 &
%		None & 2-16 &
%		The number in the word name (2-16) is pushed onto the stack.\\ \hline
%	
%	\end{tabular}
%\end{center}
%
%\subsection*{Crypto}
%\begin{center}
%	\begin{tabular}{| C{0.225\linewidth} | C{0.125\linewidth} |
%						>{\em}C{0.125\linewidth} | >{\em}C{0.125\linewidth} | C{0.35\linewidth} |}
%		\hline
%		\textbf{Word} & \textbf{Opcode} & \textbf{Input} & \textbf{Output} & \textbf{Description} \\ \hline \hline
%		
%		OP\_HASH160	& 0xA9 &
%		Data & Hash &
%		The top stack element is hashed twice: first with SHA-256 and then with RIPEMD-160.\\ \hline
%		
%		OP\_CHECKSIG & 0xAC &
%		Signature Pubkey & True / False	&
%		The transaction is hashed and the signature is verified for this hash and public key. The result is pushed onto the stack.\\ \hline
%		
%		OP\_CHECK- MULTISIG &	0xAE &
%		OP\_0 sig1 ... m pub1 ... n & True / False &
%		For each signature and public key pair, OP\_CHECKSIG is executed. The result is pushed onto the stack.\\ \hline
%
%	\end{tabular}
%\end{center}
%\vspace{10pt}
%
%\subsection*{Other}
%\begin{center}
%	\begin{tabular}{| C{0.225\linewidth} | C{0.125\linewidth} |
%						>{\em}C{0.125\linewidth} | >{\em}C{0.125\linewidth} | C{0.35\linewidth} |}
%		\hline
%		\textbf{Word} & \textbf{Opcode} & \textbf{Input} & \textbf{Output} & \textbf{Description} \\ \hline \hline
%		
%		OP\_DUP & 0x76 &
%		Data & Data Data &
%		Duplicates the top stack item.\\ \hline
%		
%		OP\_EQUAL & 0x87 &
%		Data1 Data2 & True / False &
%		Returns \emph{true} if the top two stack elements are exactly equal, \emph{false} otherwise. The result is pushed onto the stack.\\ \hline
%
%		OP\_EQUAL- VERIFY & 0x88 &
%		Data1 Data2 & True / False & 
%		Same as OP\_EQUAL, but marks transaction as invalid if it fails.\\ \hline
%
%		OP\_RETURN & 0x6A &
%		None & None &
%		Marks transaction as invalid.\\ \hline
%
%	\end{tabular}
%\end{center}
%\vspace{10pt}

\end{appendices}